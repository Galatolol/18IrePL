
\documentclass[10pt,a4paper,twocolumn]{article}
\usepackage[utf8]{inputenc}
\usepackage{amsmath}
\usepackage{amsfonts}
\usepackage{amssymb}
\usepackage{graphicx}
\usepackage{hyperref}
\usepackage[parfill]{parskip}
\usepackage{makecell}
\usepackage{multirow}
\usepackage[table]{xcolor}


\title{18IrePL\\ \large{Rules differences with 18Ireland}}
\author{Galatolol}
\date{\today}


\newcommand{\Example}{\textbf{Example:}}
\newcommand{\Line}{\noindent\rule{7.8cm}{0.4pt}}

\begin{document}
\maketitle

\section{Introduction}
This is my take on \href{https://boardgamegeek.com/boardgame/196217/18ireland}{Ian Scrivins' 18Ireland}. I wanted to make it more focused on mergers and to cut some chrome that felt to me unnecessary. The biggest changes are: 
\begin{itemize}
\item no secondary rail network
\item trains are one-sided and they are direclty removed from the game upon rusting
\item 10-share companies cannot be started directly, they must form as a result of a merger
\end{itemize}

The game is set in modern-day Poland. Historically it is completely inaccurate. All companies are imaginary, however major companies are named after real railroad connections and companies.

This document covers the rules differences between the two games.


%\adforn{21}\quad\adforn{11}\quad\adforn{49}

\section{Definitions}
\begin{description}
	\item [Minor company] 5-share company. There is 16 of them.
	\item [Major company] 10-share company. There is 7 of them.
\end{description}

\section{Setup}

\begin{center}
  \begin{tabular}{ | c | c | c | }
    \hline
    \textbf{Players} & \textbf{Starting capital} & \textbf{Cert. limit} \\ \hline
    3 & 360 & 16 \\ \hline
    4 & 270 & 12 \\ \hline
    5 & 216 & 10 \\ \hline
    6 & 180 & 8 \\ 
    \hline
      \multicolumn{3}{|l|}{\textbf{Bank size:} 4000} \\
 	\hline
  \end{tabular}
\end{center}


\subsection{Offboard connections}
Randomly choose two offboard connections (red tiles) and place them in corresponding hexes. The rest of the offboard connections won't be used in the game. No rail can point towards an empty offboard connection slot.

Minor A and Minor B start on the chosen red tiles. To determine which is which, start at Germany and go counter-clockwise: the first red tile you meet is where Minor A starts whilst Minor B starts on the second. Place their stations near the tile as a reminder.

\textit{\Example~Germany—A, Slovakia—B;\\Ukraine—A, Russia B.}

Offboard connection may be used by any company as starting or ending stop of its train's run. The same train cannot visit more than once the same offboard connection. Only designed companies (Minor A or Minor B) may ever place a station in offboard connection. However, upon merging, this home stations is replaced by the one of the new major company in the normal way.

\subsection{Minor companies}
Set aside the company charter of the Minor 8. Shuffle the remaining minor company charters face down. Randomly remove from the game two of them. The remaining charters are placed face up in a pile. Players may consult their order at any time. 

\subsection{Private companies}
Prepare a waterfall auction: lay all private companies in order of their costs.

\subsection{+10 tokens}
Put a +10 marker on marked hexes. Cities located in those hexes have their value increased by 10. The markers aren't removed when a hex is upgraded.

\textbf{Concerned hexes: D10, D14, F4, G17, H18}


\section{Stock round}

\subsection{Starting minor companies}
In yellow phase only yellow par values are available. In green phase green par values are also avaiable. In brown phase brown par values are also available.

\textit{\Example~Bid of 280. In yellow phase the par value would be 100, whilst in green and brown phase it would be 122.}

\subsection{Starting major companies}
Major companies may only be formed as a result of a merger. They cannot be started directly in a stock round.


\section{Operating round}

\subsection{Track placement}
Companies may lay or upgrade only one track per Operating Round. \textbf{Minor 6} and \textbf{Minor 10} are exception to this rule: they may lay up to two yellow tracks on their first Operating Round (however they cannot combine track upgrade and additional track lay).


\subsection{Trains}
Trains are ``classic" N-trains. They can reach up to N stops. A stop constitues of city, town, offboard connection, or port.

4D-train reaches up to 4 stops and doubles their value.

Trains are one-sided. Once they rust, they are immediately removed from the game.

\subsubsection{Excess trains}

If at any time, a com-pany  should  find  itself with  more  trains  than  the  current train limit, the excess trains are immediately removed from the game  without  compensation.


\section{Game end}

In most cases the last Merger Round is replaced with a third Operating Round.

In case of more than one end game conditions being active, the one that ends the game faster is in effect.

\subsection{A company reaches 260 value}
Finish the current Operating Round.

\subsection{Bank runs out of money}
Finish the current set of Stock Round + Operating Rounds \textbf{plus one additional Operating Round} (so 3 Operating rounds in total).

\subsection{First 4D-train purchased}
Finish the current set of Operating Rounds, then play Stock Round and \textbf{three Operating Rounds}.

\subsection{Bankruptcy}
If a player goes bankrupt, they are removed from the game. All their companies close. Minority shareholders receive the final value (if any) of their shares from the bank.

Play continues until the end of the current set of Operating Rounds \textbf{plus one additional Operating Round} (so 3 Operating rounds in total). 

More than one player may go bankrupt. At any time, if only one player is left in the game, that player wins immediately.

\subsection{Obvious winner}
At any time, if all players agree which one of them has already won the game, that player wins immediately.

\section{Private companies}

All private companies other than Warsaw-Radom railroad close at the beginning of the phase 5. They do not close upon executing their power.

\subsection{Wrocław-Oława Railroad}

\textbf{Cost:} 20 ~~\textbf{Revenue:} 5

No special power.

\subsection{Industrialisation of Łódź}

\textbf{Cost:} 40 ~~\textbf{Revenue:} 10

Owning company may place a +20 token in Łódź (G13). For all companies the value of the city is increased by 20. The token isn't removed when \textit{Industrialisation of Łódź} closes or when the hex is upgraded. 

\subsection{Bridge Company}

\textbf{Cost:} 60 ~~\textbf{Revenue:} 15

Owning company ignores all terrain costs (water and mountain).

\subsection{Wilhelmsbahn}

\textbf{Cost:} 80 ~~\textbf{Revenue:} 20

When Wilhelmsbahn is purchased by a company, that company may immediately place a yellow track, following normal track placement rules.

Once per game, in track placement step of its operting round, owning company may pay \$20 to place a yellow track, following normal track placement rules, in addition ot its normal track placement. 

It is possible to place those two additional yellow tracks in one round. They may be directly upgraded by a normal track placement action.

\subsection{Baltic Shipping}

\textbf{Cost:} 100 ~~\textbf{Revenue:} 25

Owning company may place a \$40 token in one of the ports (B4, E1, G3). This forms a stop that only trains of the owning company may access. The token isn't removed when \textit{Baltic Shipping} closes.

\subsection{Warsaw-Radom Railroad}

\textbf{Cost:} 100 ~~\textbf{Revenue:} 0

Owning player immediately takes the presidency of Minor 8, places its home station (I11), and puts the winning bid into its treasury. Par price is the highest yellow par value that is not more than half the bid. Then Minor 8 buys a train. \textit{Warsaw-Radom Railroad} closes immediately.

\section{Acknowledgements}

The game belongs to 18xx games family started by Francis Tresham's 1829. The game is based on Ian Scrivins' 18Ireland.

I made the game files using Christopher Giroir's \href{https://github.com/kelsin/18xx}{excellent tool for prototyping 18xx games}, JC Lawerence's \href{https://github.com/clearclaw/xxpaper}{xxpapers}, GIMP, and Latex.

Cover image (railroad bridge in Tczew): \url{https://upload.wikimedia.org/wikipedia/commons/8/88/Weichselbr\%C3\%BCcke_Dirschau.JPG}

Stock market image (main station in Gdańsk): \url{https://upload.wikimedia.org/wikipedia/commons/7/79/Estaci\%C3\%B3n_de_FFCC\%2C_Gdansk\%2C_Polonia\%2C_2013-05-20\%2C_DD_01.jpg}

\twocolumn[
  \begin{@twocolumnfalse}
    \maketitle
    \section{Trains and phases}
\begin{center}
  \begin{tabular}{ | c | c | c | c | c | c |}
    \hline
    \textbf{Phase} & \textbf{Cost} & \textbf{\#} & \textbf{Train limit} & \textbf{Rust} & \textbf{Notes} \\ \hline
    \cellcolor{yellow}2 & 80 & 7 & 2/- & - & Minor companies may open.  Yellow par values available.\\ \hline
    \cellcolor{green!50}3 & 180 & 4 & 2/4 & - & \multirow{2}{*}{\makecell{Yellow and green par values available.\\ Major companeis may form by merger. May buy private companies.}} \\ \cline{1-5}
    \cellcolor{green!50}4 & 300 & 3 & 2/3 & 2 &  \\ \hline
    \cellcolor{brown}5 & 440 & 2 & \multirow{3}{*}{1/2} & - & \multirow{2}{*}{\makecell{Yellow, green, and brown par values available.\\Private companies close.\\Companies may trade in trains.}} \\ \cline{1-3}\cline{5-5}
    \cellcolor{brown}6 & 550 & 2 & & 3 &  \\ \cline{1-3}\cline{5-5}
    \cellcolor{brown}4D & 700* & $\infty$ & & 4 &  \\ \hline
  \end{tabular}
\end{center}
*possible to trade in a train and pay the difference
  \end{@twocolumnfalse}
]



\end{document}